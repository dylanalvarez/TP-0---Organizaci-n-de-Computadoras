%%%%%%%%%%%%%%%%%%%%%%%%%%%%%%%%%%%%%%%%%%%%%%%%%%%%%%%%%%%%%%%%%%%%%%%%%%%%%%%
% Definición del tipo de documento.                                           %
% Posibles tipos de papel: a4paper, letterpaper, legalpapper                  %
% Posibles tamaños de letra: 10pt, 11pt, 12pt                                 %
% Posibles clases de documentos: article, report, book, slides                %
%%%%%%%%%%%%%%%%%%%%%%%%%%%%%%%%%%%%%%%%%%%%%%%%%%%%%%%%%%%%%%%%%%%%%%%%%%%%%%%
\documentclass[a4paper,10pt]{article}


%%%%%%%%%%%%%%%%%%%%%%%%%%%%%%%%%%%%%%%%%%%%%%%%%%%%%%%%%%%%%%%%%%%%%%%%%%%%%%%
% Los paquetes permiten ampliar las capacidades de LaTeX.                     %
%%%%%%%%%%%%%%%%%%%%%%%%%%%%%%%%%%%%%%%%%%%%%%%%%%%%%%%%%%%%%%%%%%%%%%%%%%%%%%%

% Paquete para inclusión de gráficos.
\usepackage{graphicx}
\usepackage{float}

% Paquete para definir la codificación del conjunto de caracteres usado
% (latin1 es ISO 8859-1).
\usepackage[utf8]{inputenc}

% Paquete para definir el idioma usado.
\usepackage[spanish]{babel}

\begin{document}
% Título principal del documento.
\title{\textbf{Trabajo Practico 0}}

% Información sobre los autores.
\author{	Dylan Gustavo Alvarez Avalos, \textit{Padrón Nro. 98225}                     \\
            \texttt{ dylanalvarez1995@gmail.com }                                              \\
            Gerstner Facundo, \textit{Padrón Nro. 96255}                     \\
            \texttt{ facugerstner_29@hotmail.com }                                              \\
       }
\date{}

% Quita el número en la primer página.
\thispagestyle{empty}


\section{Introducción}
  
   El objetivo del presente trabajo practico fue familiarizarse con las herramientas de software que serán utilizadas en el curso. Esto se logró 
   a través de la implementación del algoritmo conocido como ``Criba de Eratóstenes", sobre un entorno de desarrollo simulado 
   mediante diversas herramientas. Se utilizo GXEmul para simular una maquina MIPS, corriendo el SO NetBSD.


\section{Implementación}
    El código fuente fue escrito en la maquina host y luego fue enviado para la compilación y ejecución a la maquina guest. Ambas fueron conectadas vía ssh y el pasaje de archivos se llevo a cabo mediante scp.

    \subsection{Consideraciones}
        Se tomo como rango valido de funcionamiento, desde 2 hasta el valor representado por la constante LONG\_MAX disponible en la librería de c limits.h. Esta constante define el máximo valor para un entero largo (long int).\\
        El programa retorna 0 si la ejecución fue exitosa y -1 ante algún error.
    \subsubsection{Proceso de compilación}
        El algoritmo implementado fue compilado con la versión de gcc incluida en el sistema operativo NetBSD, con la siguiente instrucción:\\
    \begin{verbatim}
        root@:~# gcc -Wall -Werror -std=c99 -pedantic -o erat main.c
    \end{verbatim}
    Se utilizaron dichas directivas con el objetivo de respetar el estándar 99 de c y evitar warnings. 


\section{Casos de prueba}

Se corrió el algoritmo implementado en lenguaje c, sobre GXemul con los valores -5, 1, 10, 50, y 100, con el objetivo de probar el programa y validar su correcto funcionamiento sobre la arquitectura emulada. A continuación se muestra el comando corrido en cada prueba

    \subsubsection{N=-5}
    \begin{verbatim}
       (echo -5) | ./erat 
    \end{verbatim}
    
    El programa retornó un error de fuera de rango ya que -5 no esta dentro del rango definido. El error retornado fue:\\
    \begin{verbatim}
        The number is out of range (2 to 9223372036854775807)
    \end{verbatim}
    
    \subsubsection{N=1}
    \begin{verbatim}
       (echo 1) | ./erat 
    \end{verbatim}

    Nuevamente se obtuvo el mismo error ya que se sigue pasando un argumento que no se encuentra en el rango permitido.\\
    
    \subsubsection{N=10}
    \begin{verbatim}
       (echo 10) | ./erat 
    \end{verbatim}
    
    En este caso 10 esta dentro del rango e imprimió por pantalla los números primos correctos (2, 3, 5, y 7).\\
    
   \subsubsection{N=50}
    \begin{verbatim}
       (echo 50) | ./erat 
    \end{verbatim}
    
    Nuevamente, el parámetro pasado, esta dentro del rango y el programa funciono correctamente arrojando los resultados correctos.
    
   \subsubsection{N=100}
    \begin{verbatim}
       (echo 100) | ./erat 
    \end{verbatim}

    Finalmente, también respondió de manera correcta ante el ultimo caso de prueba.
    
\section{Conclusiones}

    A través de la implementación de un algoritmo sencillo, logramos familiarizarnos con las herramientas de trabajo del curso, ya que la resolución del trabajo implicaba la instalación y configuración de las mismas, además de la ejecución del algoritmo bajo el entorno preparado. También implicó aprender y entender el pasaje de archivos entre dos sistemas operativos mediante las herramienta ssh, ya que el código fue escrito en la maquina host (ubuntu 16.04 64 bits) y luego se debió mover al sistema operativo guest (MIPS con NetBSD). Cabe destacar que no se notaron diferencias con respecto a la arquitectura y el sistema operativo nativo sobre el que se emulaba. En trabajos que requieran un algoritmo mas complejo se esperan observar diferencias en distintos aspectos, por ejemplo el tamaño de los tipos en la arquitectura MIPS o los tiempos de compilación y ejecución, que deberían ser mas lentos por tratarse de un entorno emulado.


\section{Apéndice}
A continuación se presenta el código escrito para el presente trabajo.\\
  \begin{verbatim}
#include <stdio.h>
#include <stdbool.h>
#include <string.h>
#include <stdlib.h>
#include <getopt.h>
#include <errno.h>
#include <limits.h>

#define SUCCESS 0
#define ERROR 1
#define VERSION 1.0
#define NUMBER_OF_DIGITS_IN_LONG_MAX 20

void print_info() {
    printf("%s\n%s\n%s\n%s\n",
           "Usage:",
           "erat -h",
           "erat -V",
           "erat [options] N");
    printf("%s\n%s\n%s\n%s\n",
           "Options:",
           "-h, --help Prints usage information",
           "-V, --version Prints version information",
           "-o, --output Path to output file");
    printf("%s\n%s\n",
           "Examples:",
           "erat -o - 10");
}

void print_version() {
    printf("%s%.2f\n", "Version: ", VERSION);
}

void erat(long upper_limit, FILE *file) {
    int i, j;
    for (i = 2; i <= upper_limit; i++) {
        bool is_prime = true;
        for (j = 2; j < i && is_prime; j++) {
            if (i % j == 0) {
                is_prime = false;
            }
        }
        if (is_prime) {
            fprintf(file, "%d\n", i);
        }
    }
}

void print_out_of_range_error() {
    fprintf(stderr, "The number is out of range (%d to %li)\n", 2, LONG_MAX);
}

int main(int argc, char **argv) {
    int argument;
    char input[NUMBER_OF_DIGITS_IN_LONG_MAX];
    FILE *output = NULL;
    char *lastChar;
    long number;
    bool readFromStdin = false;

    static struct option options[] =
        {
            {"help",    no_argument,       NULL, 'h'},
            {"version", no_argument,       NULL, 'V'},
            {"output",  required_argument, NULL, 'o'},
            {NULL,      no_argument,       NULL, 0}
        };

    argument = getopt_long(argc, argv, "hVo:", options, NULL);

    switch (argument) {
        case 'h':
            print_info();
            return SUCCESS;
        case 'V':
            print_version();
            return SUCCESS;
        case 'o':
            if (strcmp(optarg, "-") != 0) {
                output = fopen(optarg, "wb");
                if (!output) {
                    fprintf(stderr,
                            "The file '%s' could not be created/opened\n",
                            optarg);
                    return ERROR;
                }
            }
            if (argc < 4) {
                fprintf(stderr,
                        "The number is missing\n");
                return ERROR;
            }
        default:
            if (argc < 4) {
                if (!fgets(input, NUMBER_OF_DIGITS_IN_LONG_MAX, stdin)) {
                    print_out_of_range_error();
                    return ERROR;
                }
                readFromStdin = true;
            }
            number = strtol(readFromStdin ? input : argv[3], &lastChar, 10);
            if (errno == ERANGE && (number == LONG_MIN || number == LONG_MAX)) {
                print_out_of_range_error();
                return ERROR;
            }
            if ((readFromStdin ? input : argv[3]) == lastChar) {
                fprintf(stderr,
                        "The number is invalid\n");
                return ERROR;
            }
            if (number < 2) {
                print_out_of_range_error();
                return ERROR;
            }
            erat(number, output ? output : stdout);
            if (output) { fclose(output); }
            return SUCCESS;
    }
}

  \end{verbatim}
  
\end{document}
